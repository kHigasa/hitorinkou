\chapter{Introduction} % (fold)
\label{cha:Introduction}

\section{Simple Experimental Facts} % (fold)
\label{sec:Simple Experimental Facts}
% section Simple Experimental Facts(end)

\section{Phenomenological Theories} % (fold)
\label{sec:Phenomenological Theories}
\subsection{Gorter-Casimir Model} % (fold)
\label{sub:Gorter-Casimir Model}
(1-6). 
\begin{align}
    \pdv{F}{x} = \frac{1}{2}\frac{1}{\sqrt{x}}\qty(-\frac{1}{2}\gamma T^2)-1(-\beta) &= 0 \notag\\
    \frac{\gamma T^2}{4}\frac{1}{\sqrt{x}} &= \beta \notag\\
    \therefore\sqrt{x} &= \frac{\gamma T^2}{4\beta} \notag\\
    \therefore x &= \uwave{\qty(\frac{\gamma}{4\beta})^2T^4} \notag
\end{align}

(1-8). 
\begin{align}
    H_c^2(T) &= 8\pi(F_n(T)-F_s(T)) = 8\pi(F(1,T)-F(0,T)) \notag\\
    &= 8\pi(f_n(T)-f_s(T)) = 8\pi\qty(-\frac{1}{2}\gamma T^2+\beta) \notag\\
    &= 8\pi\beta\qty(1-\frac{\gamma}{2\beta}T^2) \equiv \uwave{H_0\qty(1-\qty(\frac{T}{T_c}^2))} \notag
\end{align}

(1-9). 
\begin{align}
    C_{es}(T) &= -T\qty(\pdv[2]{F}{T}) = -T\sqrt{x}(-\gamma) = \gamma T\sqrt{x} \notag\\
    &= \uwave{\gamma T_c\qty(\frac{T}{T_c})^3} \because \text{Eq.(1-6)} \notag
\end{align}
% subsection Gorter-Casimir Model(end)

\subsection{The London Theory} % (fold)
\label{sub:The London Theory}
% subsection The London Theory(end)

\subsection{F.London's Justification of the London Theory} % (fold)
\label{sub:F.London's Justification of the London Theory}
% subsection F.London's Justification of the London Theory(end)

\subsection{Pippard's Nonlocal Generalization of the London Theory} % (fold)
\label{sub:Pippard's Nonlocal Generalization of the London Theory}
% subsection Pippard's Nonlocal Generalization of the London Theory(end)
\subsection{Ginsburg-Landau Theory} % (fold)
\label{sub:Ginsburg-Landau Theory}
(1-43). 
\begin{align}
    f(T) &= a\qty(-\frac{a}{b})+\frac{1}{2}b\qty(-\frac{a}{b})^2 \notag\\
    &= -\frac{a^2}{b}+\frac{a^2}{2b} = \uwave{-\frac{a^2(T)}{2b(T)}} \notag
\end{align}

(1-44). 
\begin{align}
    \frac{\lambda^2(0)}{\lambda^2(T)} &= \frac{n_s^2(T)}{n_s^2(0)} = \frac{\abs{\Psi_e(T)}^2}{\abs{\Psi_e(0)}^2} \because \text{Eq.(1-38)} \notag\\
    &= \abs{\Psi_e(T)}^2\because n=n_s\qfor T=0 \notag\\
    \therefore&\uwave{\abs{\Psi_e(0)}=1} \notag
\end{align}

(1-48). 
\begin{align}
    &\pdv{f}{\Psi(\vb{r})} = 0 \notag\\
    \Rightarrow 0 &= -\frac{\hbar^2}{2m^*}\qty(\nabla +\frac{\mathrm{i}e^*}{\hbar c}A(\vb{r}))2\Psi(\vb{r}) \notag\\
    0 &= -\frac{H_c^2(T)}{4\pi m^*}\frac{\lambda^2(T)}{\lambda^2(0)}\qty(1-\frac{\lambda^2(T)}{\lambda^2(0)}\abs{\Psi(\vb{r})}^2)2\Psi(\vb{r}) \notag\\
    \therefore&\uwave{\text{Eq.(1-48)}} \notag
\end{align}

(1-51). 

(1-52). 
% subsection Ginsburg-Landau Theory(end)
% section Phenomenological Theories(end)
% chapter Introduction(end)
